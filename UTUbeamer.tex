% Originally by Riku Klén (2013)
% Modified by Kaisa Joki (2019)
% Modified by Teemu Daniel Laajala (2023)
% Beamer slide template based on Warsaw theme for use in UTU, particularly Department of Mathematics and Statistics

\documentclass{beamer} 

\mode<presentation>

\usepackage{amssymb,amsthm,amsmath} % ams
\usepackage[T1]{fontenc} % Font coding
\usepackage[utf8]{inputenc} % For Scandinavian letters etc
\usefonttheme{professionalfonts}

\usepackage{booktabs}
\usepackage[english]{babel} % Language
\usepackage{pstricks} % Required For command \newrgbcolor
\usepackage{xcolor} % Required for command \colorbox

\usetheme{Warsaw} % Beamer-style: Warsaw
\useoutertheme[subsection=false]{smoothbars}

\usepackage{multirow}
\usepackage{colortbl}
\usepackage{hhline}

\usefonttheme{professionalfonts}

\usepackage{graphicx}
\usepackage{subcaption}

\setlength{\arrayrulewidth}{0.5pt}

\definecolor{plum(traditional)}{rgb}{0.56, 0.27, 0.52}
\definecolor{pansypurple}{rgb}{0.47, 0.09, 0.29}
\definecolor{palatinatepurple}{rgb}{0.41, 0.16, 0.38}
\definecolor{darkraspberry}{rgb}{0.53, 0.15, 0.34}
\definecolor{spring green}{rgb}{0.000000,1.000000,0.498039}
\definecolor{mypink1}{rgb}{0.858, 0.188, 0.478}

\newrgbcolor{turku}{0.44 0.16 0.39} % University of Turku color
\setbeamertemplate{blocks}[rounded][shadow=true] % Style settings
\setbeamercolor{structure}{fg=palatinatepurple!70!black} % Style settings
\setbeamercolor{block title}{bg=palatinatepurple!85!black} % Style settings



\setbeamertemplate{theorems}[numbered] % Number theorems

\pgfdeclareimage[width=128mm]{logo}{UTUhorizontal2012.png} % Define picture as a horizontal bar and logo
\logo{\vspace{-4mm}\pgfuseimage{logo}\hspace{-1mm}} % UTU logo, requires the image file "UTUhorizontal2012.png" (or similar)

\title[Title visible on the lower-right corner] % Short header, visible on a lower corner
{ Full title \\ on multiple \\ lines} % Title to the first page

\author[\insertframenumber / \inserttotalframenumber \hspace{4cm} John Doe et al.] % Short author(s) on bottom
{ John Doe }% Authors on the first slide

%\%institute[] % Instituutin lyhenne
%{$^1$University of Turku \\ $^2$ and maybe some other place as well} % Instituutti

\date[] % Short for date
{Date and \\ place \vspace{-0.25cm}} % Date on the first slide





% Some example environments used by KJ
\theoremstyle{plain}
%\newtheorem{Theo}[Def]{THEOREM}
\newtheorem{lause}[equation]{Lause}
\theoremstyle{definition}
\newtheorem*{esimerkki}{Esimerkki}

\setbeamercolor{firb}{fg=black,bg=white!80!palatinatepurple}
\newenvironment{flatbox}[1]{\begin{beamercolorbox}[sep=1em,wd=1.0\textwidth]{#1}}{\end{beamercolorbox}}

\theoremstyle{definition}
\newtheorem{Alg}{ALGORITHM}[section]

% Custom commands from KJ
\newcommand{\x}{{\boldsymbol x}}           % Bold-face for vector x
\newcommand{\y}{{\boldsymbol y}}           % Bold-face for vector y

\newcommand{\bxi}{{\boldsymbol \xi}}           % vektori \xi lihavoitu
\newcommand{\bd}{{\boldsymbol d}}           % vektori d lihavoitu
\newcommand{\bv}{{\boldsymbol v}}           % vektori v lihavoitu
\newcommand{\be}{{\boldsymbol e}}           % vektori e lihavoitu
\newcommand{\ba}{{\boldsymbol a}}           % vektori e lihavoitu
\newcommand{\bu}{{\boldsymbol u}}           % vektori e lihavoitu
\newcommand{\err}{e}                        % error function

\newcommand{\bzero}{{\boldsymbol 0}}           % vektori 0 lihavoitu
\newcommand{\bM}{{\boldsymbol M}}           % matriisi M lihavoitu
\usepackage{setspace}

\newcommand{\R}{\mathbb{R}}
\newcommand{\B}{\mathbb{B}}
\newcommand{\Bn}{{\mathbb{B}^n}}
\newcommand{\Rn}{{\R^n}}
\newcommand{\comment}[1]{}
\DeclareMathOperator{\st}{s.t.}             % In optimization abbreviation for "subject to"


\begin{document}

\setbeamertemplate{navigation symbols}{} % Remove navigation symbols

\begin{frame}
  \titlepage
  
 { \begin{spacing}{0.5}
\begin{tiny}
  This work was financially supported by the Foo and Bar organizations 
  \end{tiny}\end{spacing} } % Funding statement
  
\end{frame}

\begin{frame}
   \frametitle{Outline}
   \tableofcontents
\end{frame}

\section{Introduction and Motivation}
\subsection{Step one}

\begin{frame}{The very first real slide}	
\begin{small}

\begin{itemize}
\item First thing
\item Second thing
\item Third thing
{\LARGE 
\begin{itemize}
\item Third sub-thing one
\item Third sub-thing two
\end{itemize}}

\end{itemize}\vspace{.15cm}

\end{small}
\end{frame}

\subsection{Step two}

\begin{frame}{The second slide}
\begin{small}


\begin{block}{A nice little block}\vspace{-0.3cm}
	\begin{align*}
		\begin{cases}
		\text{minimize}\quad\, & f(\x) \\
		\text{subject to} & \x \in \R^n
		\end{cases}
	\end{align*}
\end{block}

\begin{itemize}

\item Insight into the equation

\item Deeper insight
\begin{itemize}
  \item Open up insight
  \item Open insight even more
\end{itemize} \vspace{0.3cm}

\item[$\Rightarrow$] {\bf \color{palatinatepurple!80!black} Heureka!}

\end{itemize}

\end{small}
\end{frame}

\section{Methodology}
\subsection{What are we using}

\begin{frame}{Thing we use first}	
\begin{small}


\begin{itemize}
\item {\bf \em \color{palatinatepurple!80!black} A function} $f:\R \rightarrow\R$ is assumed to
	\begin{itemize}
	  \item[--] Assumption 1
	  \item[--] Assumption 2
	  \item[--] Assumption 3
	\end{itemize}	
\item And so forth
\item And even further
\end{itemize}

\end{small}
\end{frame}


\begin{frame}{Also used}	
\begin{small}

\begin{block}{I am also a block}
	... and here I have my important stuff.
\end{block}

\begin{block}{}
	Followed by a block without a title.
\end{block}

\vspace{1cm}
Followed by free text after some padding.

\end{small}
\end{frame}


\section{End}

\begin{frame}{\text{ }}


\begin{center}
\begin{LARGE}
\textbf{Thank you for your attention!}
\end{LARGE}
\end{center}

\end{frame}


\end{document}


